\documentclass{beamer}

\usepackage{mathtools}
\usepackage{amssymb}
\usepackage{amsthm}
\usepackage{cmap}
\usepackage{fontspec}
\usepackage{unicode-math}
\defaultfontfeatures{Ligatures=TeX}
\setmainfont[Extension=.otf,UprightFont=*-regular,ItalicFont=*-italic,BoldFont=*-bold,BoldItalicFont=*-bolditalic]{texgyretermes}
\setsansfont[Extension=.otf,UprightFont=*-regular,ItalicFont=*-italic,BoldFont=*-bold,BoldItalicFont=*-bolditalic]{texgyreadventor}
\setmathfont{xits-math.otf}
\usepackage[protrusion=true,expansion=true,verbose=true]{microtype}

\usetheme{Berkeley}
\usecolortheme{crane}

\title[LaTeX]{LaTeX: a crash course}
\author{Niko Ilomäki}
\institute{Matematiikan yö k2014}
\date{\today}

\begin{document}

\begin{frame}
\titlepage
\end{frame}

\begin{frame}{LaTeX (1)}
LaTeX...
\begin{itemize}
\pause \item is a document preparation system.
\pause \item offers high-quality typesetting % industry standard
\pause \item differs a lot from e.g. Microsoft Word.
\end{itemize}
\end{frame}

\begin{frame}{LaTeX (2)}
LaTeX...
\begin{itemize}
\pause \item separates content and structure to some extent. % compare to XML
\pause \item is based on an earlier system, TeX.
\pause \item is still developed.
\end{itemize}
\end{frame}

\begin{frame}{LaTeX compilers}
There are several LaTeX compilers. We'll focus on the ones that compile .tex to .pdf.
\begin{itemize}
\pause \item pdflatex
\pause \item xelatex
\pause \item lualatex
\end{itemize}
\end{frame}

\begin{frame}{Writing math with LaTeX}
Math can be written in several modes in LaTeX. The most common ones are:
\begin{itemize}
\pause \item Inline mode: \$..\$ or \textbackslash(..\textbackslash)
\pause \item Display mode: \textbackslash[..\textbackslash] (not recommended: \$\$ .. \$\$)
\pause \item Align mode: \textbackslash begin\{align*\} .. \textbackslash end\{align*\}
\end{itemize}
\end{frame}

\end{document}
